\section{クラス}
機械学習は,パラメータを少しずつ調整しながら,数多くのモデルを比較検討したりするため,管理すべき変数の数は膨大となり,プログラムの見通しが著しく悪くなることが多い.機械学習のモデルは,モデルを決めて,ハイパーパラメータを決めて,データを入れて,何らかのアルゴリズムでパラメータ等を推定し,決定されたモデルに別の新しいデータを入れて予測をすることである.パラメータ,モデル式,学習アルゴリズム等をひとまとめにしてモデルという1つのオブジェクトとして管理できるようにできれば,モデル1つに対して紐づいているたくさんの変数を注意深く管理していく必要はなくなる.

クラスとは,「各種変数や関数をひとまとめにしたオブジェクトを作るための生成器」であり,また,「クラスにパラメータやデータを与えることで具体的に生成した各種変数や計算式をひとまとめにしたオブジェクト」をインスタンスという.なお,インスタンス内で使われる変数をインスタンス変数といい,インスタンス内で使われる関数をメソッドという.

クラスを使えば,各モデルの区別はインスタンス名で区別され,インスタンス変数の名前はどのモデルでも同じ名前を使うことが可能となる.

\subsection{クラス・インスタンスの作成}

クラスの定義は以下のようにして行う.\texttt{self.<val\_name1>}などの形をしているものがインスタンス変数である.
\begin{gram}[クラスの定義] 
\begin{lstlisting}
class <class_name>:
	def __init__(self, <param_1>=a, <param_2>=b,....): #(1)
		self.<val_name1> = <param_1>
		self.<val_name2> = <param_2>
		...
	
	def <method1_name>(self, <object>, <object>,....): #(2)
		<expr>
		<expr>
		...
	
	def <method2_name>(self, <object>, <object>,....): #(3)
		<expr>
		<expr>
		...
		return <expr>
\end{lstlisting}
\end{gram}
クラスを定義できたら,そのクラスからインスタンスを生成できる.
\begin{gram}[インスタンスの生成] 
\begin{itemize}
\item \texttt{<instance>=<class\_name>(<param\_1>=x,<param\_2>=y,...)}: \texttt{<class>}にパラメータを与えてインスタンスを作成しそれを\texttt{<instance>}の名前のインスタンスオブジェクトとして格納する.
\end{itemize}
\end{gram}
インスタンス生成を行うと,自動的に\texttt{\#(1)}に記述している\texttt{\_\_init\_\_}メソッドが実行され,インスタンス変数にインスタンス生成時で指定したパラメータの代入処理が行われる(メソッド名を\texttt{\_\_init\_\_}としないと,インスタンス生成時の自動実行処理は行われないので注意).なお,\texttt{\_\_init\_\_}メソッドの引数で\texttt{<param\_1>=a}というように書かれているものは,インスタンス生成時に\texttt{<param\_1>}に何も数値を指定しない場合,デフォルト値として\texttt{a}をインスタンス変数に代入するようにしたいときの書き方である.また,\texttt{self}とはインスタンス自身を表す.インスタンスの名前はクラスに反映されないので,\texttt{self}という名前で取り扱うという意味である.

インスタンス変数とインスタンスの外から与えたデータから計算を行うメソッドは,返り値を返さないもの\texttt{\#(2)}と返すもの\texttt{\#(3)}の2種類を記述できる.\texttt{\#(2)}はインスタンス変数を変化させたいときに行うもので,特段クラスの外に新しい変数として取り出す必要のないときに用いられる.\texttt{\#(3)}は実行結果を返すので,最終的な計算結果をアウトプットしたいときなどに用いられる.また,メソッドの引数のうち,\texttt{self}はインスタンス内の全てのインスタンス変数を渡すという意味で,インスタンスの外の変数も渡す場合は普通の関数のように記述する.

インスタンス変数は,クラスの外からでも\texttt{<instance>.<val\_name>}と指定すればアクセスできる.

\begin{cod}[\texttt{cls1.py}] \\
\texttt{Rectangle}という長方形のクラスを作成した例.インスタンス生成時に縦横の長さを自動セットし,対角線の長さを計算するメソッド,インスタンスの外から高さを与えて体積を計算するメソッド,インスタンスの外の変数に計算結果を格納する形で書いた,周囲の長さを求めるメソッドを実装した例.複数の長方形に対する縦横の長さだったり対角線の長さだったりは,\texttt{<instance>.<val\_name>}の形で記述できるため,\texttt{<val\_name>}の部分で長方形ごとに使い分けるように名前をちょっとずつ変えたりということはしなくてもよくなるため,見通しがとても良い.
\lstinputlisting[backgroundcolor={\color[gray]{.95}}]{code/cls1.py}
\vspace{-10pt}
\begin{lstlisting}
obj1:width=5,length=12
obj2:width=3,length=4
obj1:diag=13.0
obj2:diag=5.0
obj1:vollume=600
obj2:perimeter=14
obj2:area=12.0
\end{lstlisting}
\end{cod}
\vspace{-10pt}

インスタンス変数をクラス内のメソッドで使いこなす場合,クラスの利用側でメソッドの依存関係を理解しておかなければならないことに注意する.上の例では,\texttt{<instance>.calc\_area()}は\texttt{<instance>.calc\_volume()}で生成されるインスタンス変数\texttt{<instance>.volume}と\texttt{<instance>.height}を使用して動くメソッドなので,\texttt{<instance>.calc\_volume()}を実行しないまま\texttt{<instance>.calc\_area()}を実行した場合はエラーになる.

よって,外部パッケージを使用する際も,このようなことを念頭に入れておくと,多少は楽になると思われる.

