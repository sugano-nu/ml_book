\section{行列}

\begin{defi}[行列] \\
実数の長方形配列を{\bf 行列(matrix)}と呼ぶ.すなわち,行列は一般的な形として以下のように配列された実数$a_{11},a_{12},\ldots,a_{1n},\ldots,a_{m1},a_{m2},\ldots,a_{mn} \in \mathbb{R}$の集まりである.
\begin{align*}
\begin{bmatrix}
a_{11} & a_{12} & \cdots & a_{1n} \\
a_{21} & a_{22} & \cdots & a_{2n} \\
\vdots & \vdots & \vdots & \vdots \\
a_{m1} & a_{m2} & \cdots & a_{mn} 
\end{bmatrix}	
\end{align*}
なお,上記のように$m$個の行と$n$個の列からなる行列を$m\times n$行列と呼び,$m$と$n$をこの行列の{\bf 次元(dimension)}と呼ぶ.
\end{defi}

\begin{theo}
\begin{align}
{\bm a}^T{\bm b}&= {\bm b}^T{\bm a}	\\
(k{\bm a})^T{\bm b}&= k{\bm a}^T{\bm b} \\
({\bm a}+{\bm b})^T({\bm c}+{\bm d})&= {\bm a}^T{\bm c}+{\bm b}^T{\bm c}+{\bm a}^T{\bm d}+{\bm b}^T{\bm d}
\end{align}
\end{theo}

\begin{theo}
\begin{align}
(A+B)^T&= A^T+B^T \\
(AB)^T&=B^TA^T	
\end{align}
\end{theo}

\begin{defi}
$X^T=X$を満たす行列を{\bf 対称行列}という.	
\end{defi}

\begin{theo}[1次形式,2次形式の勾配ベクトル]
\begin{align}
\nabla_{{\bm x}}({\bm a}^T {\bm x})&=\nabla_{{\bm x}}({\bm x}^T {\bm a})={\bm a}\\
\nabla_{{\bm x}}({\bm x}^T A{\bm x})&=(A+A^T){\bm x}
\end{align}
\end{theo}

\begin{theo}
$X^TX$は対称行列である.
\end{theo}