\section{numpy}

\subsection{任意の\texttt{ndarray}オブジェクトの作成(\texttt{array}関数)}

\texttt{<ndarray>}は,\texttt{<list>}から作成できる.\texttt{<list>}と\texttt{<ndarray>}の見た目には,要素間にカンマがあるかないかの違いがある.

\begin{gram} 
\begin{itemize}
\item \texttt{np.array(<list>)}: \texttt{<list>}から\texttt{<ndarray>}を作成する.
\end{itemize}
\end{gram}

\begin{cod}[\texttt{num1.py}] 
\lstinputlisting[backgroundcolor={\color[gray]{.95}}]{code/num1.py}
\vspace{-7pt}
\begin{lstlisting}
x=[1 2 3],type=<class 'numpy.ndarray'>
y=[ 4.1  5.9 -6.3  7. ],type=<class 'numpy.ndarray'>
mylist=[4.1, 5.9, -6.3, 7.0],type=<class 'list'>
\end{lstlisting}
\end{cod}
\vspace{-10pt}

\subsection{規則的な\texttt{ndarray}オブジェクトの作成(\texttt{arange}関数など)}
\texttt{<ndarray>}をリスト直打ちで作成するのは,大きい配列を作ろうとする場合は不便であるが,規則的な配列であれば,別のプログラムで書くことができる.

\begin{gram} 
\begin{itemize}
\item \texttt{np.arange(a,b,c)}: \texttt{a}から\texttt{b}の手前まで,\texttt{c}ずつ増加していく\texttt{<ndarray>}を作成する.
\item \texttt{np.arange(a)}: $0$から\texttt{a}の手前まで,$1$ずつ増加していく\texttt{<ndarray>}を作成する.
\item \texttt{np.arange(a,b)}: \texttt{a}から\texttt{b}の手前まで,$1$ずつ増加していく\texttt{<ndarray>}を作成する.
\item \texttt{np.random.randn(a)}: 標準正規分布に従う乱数から発生する\texttt{a}個の要素の\texttt{<ndarray>}を作成する.
\end{itemize}
\end{gram}

\begin{cod}[\texttt{num2.py}] 
\lstinputlisting[backgroundcolor={\color[gray]{.95}}]{code/num2.py}
\vspace{-10pt}
\begin{lstlisting}
w=[ 0  2  4  6  8 10 12 14 16 18 20 22 24 26 28]
x=[0 1 2 3 4 5 6 7 8 9]
y=[-10  -9  -8  -7  -6  -5  -4  -3  -2  -1   0]
z=[ 0.0259179   0.49422254  1.0009598  -0.42038568  0.27521985  1.64888587
  0.02695996 -0.13054361 -1.24832705  0.49654017]
\end{lstlisting}
\end{cod}
\vspace{-10pt}