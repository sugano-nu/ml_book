\section{numpy}

\subsection{\texttt{ndarray}オブジェクト}

\texttt{ndarray}オブジェクトは,\texttt{list}オブジェクトから作成できる.\texttt{list}オブジェクトと\texttt{ndarray}の見た目には,要素間にカンマがあるかないかの違いがある.

\begin{gram} 
\begin{itemize}
\item \texttt{np.array([list])}: \texttt{list}オブジェクトから\texttt{ndarray}オブジェクトを作成する.
\end{itemize}
\end{gram}

\begin{cod}[\texttt{num1.py}] 
\lstinputlisting[backgroundcolor={\color[gray]{.95}}]{code/num1.py}
\vspace{-7pt}
\begin{lstlisting}
x=[1 2 3],type=<class 'numpy.ndarray'>
y=[ 4.1  5.9 -6.3  7. ],type=<class 'numpy.ndarray'>
mylist=[4.1, 5.9, -6.3, 7.0],type=<class 'list'>
\end{lstlisting}
\end{cod}
\vspace{-10pt}

\texttt{ndarray}オブジェクトをリスト直打ちで作成するのは,大きい配列を作ろうとする場合は不便であるが,規則的な配列であれば,別のプログラムで書くことができる.

\begin{gram} 
\begin{itemize}
\item \texttt{np.arange([int1],[int2],[int3])}: \texttt{[int1]}から\texttt{[int2]}の手前まで,\texttt{[int3]}ずつ増加していく\texttt{ndarray}オブジェクトを作成する.
\item \texttt{np.arange(a)}: $0$から$a$の手前まで,$1$ずつ増加していく\texttt{ndarray}オブジェクトを作成する.
\item \texttt{np.random.randn(a)}: 標準正規分布に従う乱数から発生する$a$個の要素の\texttt{ndarray}オブジェクトを作成する.
\end{itemize}
\end{gram}

\begin{cod}[\texttt{num2.py}] 
\lstinputlisting[backgroundcolor={\color[gray]{.95}}]{code/num2.py}
\vspace{-10pt}
\begin{lstlisting}
x=[ 0  2  4  6  8 10 12 14 16 18 20 22 24 26 28]
y=[0 1 2 3 4 5 6 7 8 9]
z=[ 0.41391684 -0.27481322 -0.29993306 -0.64974345  0.63441892  0.66554782 -0.52705262  1.1350932  -0.6862161   1.15003734]
\end{lstlisting}
\end{cod}
\vspace{-10pt}