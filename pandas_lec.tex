\section{pandas}

\subsection{\texttt{DataFrame}オブジェクト}

機械学習の元となるデータは基本的に表形式データであるため,まずそれを読み込むところから始まる.pandasの\texttt{DataFrame}オブジェクトは,表形式データを格納できる.カンマ区切りのテキストファイル(\texttt{txt,csv})は,\texttt{pd.read\_csv()}で読み込める.

\begin{gram} 
\begin{itemize}
\item \texttt{pd.read\_csv([str], names=[list])}: \texttt{[str]}で指定したファイルパスのデータを読み込み\texttt{DataFrame}オブジェクトに格納する.\texttt{name}は,列名をつけるときに指定する.
\end{itemize}
\end{gram}

\begin{cod}[\texttt{pd1.py}]
ここで読み込む\texttt{ex1data1\_test.txt}は,以下のようなデータである(5行2列,カンマ区切り,列名ヘッダーなし).左はpopulation of city in 10,000s,右はprofit in \$10,000sであるため,読み込む際に列名をつけている.
\lstinputlisting{data/ex1data1_test.txt}
\lstinputlisting[backgroundcolor={\color[gray]{.95}}]{code/pd1.py}
\vspace{-10pt}
\begin{lstlisting}
df=
   population   profit
0      6.1101  17.5920
1      5.5277   9.1302
2      8.5186  13.6620
3      7.0032  11.8540
4      5.8598   6.8233
type=<class 'pandas.core.frame.DataFrame'>
\end{lstlisting}
\end{cod}
\vspace{-10pt}

読み込んだ\texttt{DataFrame}から,列\texttt{column}を取り出したいときは,\texttt{[DataFrame][column].values}メソッドを使う.\texttt{ndarray}オブジェクトで取り出されるので,その後matplotlibにデータを渡しやすい.

\begin{gram} 
\begin{itemize}
\item \texttt{[DataFrame][column].values}: \texttt{[DataFrame]}の\texttt{column}列を\texttt{ndarray}オブジェクトとして抜き出す.
\end{itemize}
\end{gram}

\begin{cod}[\texttt{pd2.py}] 
\lstinputlisting[backgroundcolor={\color[gray]{.95}}]{code/pd2.py}
\vspace{-10pt}
\begin{lstlisting}
population=[6.1101 5.5277 8.5186 7.0032 5.8598],type=<class 'numpy.ndarray'>
\end{lstlisting}
\end{cod}
\vspace{-10pt}