\section{pandas}

\subsection{カンマ区切りテキストを\texttt{DataFrame}オブジェクトに格納(\texttt{read\_csv}関数)}

機械学習の元となるデータは基本的に表形式データであるため,まずそれを読み込むところから始まる.pandasの\texttt{<DataFrame>}は,表形式データを格納できる.カンマ区切りのテキストファイル(\texttt{txt,csv})は,\texttt{pd.read\_csv()}で読み込める.また,読み込んだ表の行数と列数は,\texttt{<DataFrame>.shape}で取り出せる.また,読み込む\texttt{<DataFrame>}は巨大であることが多いので,読み込んだあとはその中身の最初の数行を表示して読み込まれているかどうかを確認することが多い.このような処理は\texttt{<DataFrame>.head(<int>)}で行うことができる.

\begin{gram} 
\begin{itemize}
\item \texttt{pd.read\_csv(<str>, names=<list>)}: \texttt{<str>}で指定したファイルパスのデータ(カンマ区切り)を読み込み\texttt{<DataFrame>}に格納する.\texttt{names}は,列名をつけるときに指定する.
\item \texttt{<DataFrame>.shape}: \texttt{<DataFrame>}の行数と列数を\texttt{(行数, 列数)}の形のタプルで出力する.ここで,列数にはindex列はカウントされないことに注意する.
\item \texttt{<DataFrame>.head(<int>)}: \texttt{<DataFrame>}の最初の\texttt{<int>}行を\texttt{<DataFrame>}で取り出す.
\end{itemize}
\end{gram}

\begin{cod}[\texttt{pd1.py}]
ここで読み込む\texttt{ex1data1.txt}は,以下のようなデータである(97行2列,カンマ区切り,列名ヘッダーなし.以下では最初の5行だけ表示している).データは,左はpopulation of city in 10,000s,右はprofit in \$10,000sを表している.ヘッダーがないので,読み込む際は列名をつけている.
\lstinputlisting{data/ex1data1_test.txt}
\lstinputlisting[backgroundcolor={\color[gray]{.95}}]{code/pd1.py}
\vspace{-10pt}
\begin{lstlisting}
df=
   population   profit
0      6.1101  17.5920
1      5.5277   9.1302
2      8.5186  13.6620
3      7.0032  11.8540
4      5.8598   6.8233
5      8.3829  11.8860
6      7.4764   4.3483
7      8.5781  12.0000
8      6.4862   6.5987
9      5.0546   3.8166
type=<class 'pandas.core.frame.DataFrame'>
low_num=97, col_num=2
\end{lstlisting}
\end{cod}
\vspace{-10pt}

\subsection{\texttt{<2d\_ndarray>}から\texttt{<DataFrame>}の作成}

\texttt{<2d\_ndarray>}から\texttt{<DataFrame>}を作成できる.このときの列名は\texttt{<list>}で指定する.

\begin{gram} 
\begin{itemize}
\item \texttt{pd.DataFrame(<2d\_ndarray>, columns=<list>)}: \texttt{<2d\_ndarray>}から\texttt{<DataFrame>}を作成する.\texttt{columns}は,列名をつけるときに指定する.
\end{itemize}
\end{gram}
\begin{cod}[\texttt{pd6.py}] 
\lstinputlisting[backgroundcolor={\color[gray]{.95}}]{code/pd6.py}
\vspace{-10pt}
\begin{lstlisting}
df=
   col1  col2  col3
0     1     2     3
1     4     5     6
2     7     8     9
3    10    11    12
\end{lstlisting}
\end{cod}
\vspace{-10pt}

\subsection{列の追加}

機械学習では,新たに特徴量をデータに追加することが頻繁に行われる.全ての行の数値が同じ列を追加するには,新しい列を指定して数値を代入する文を書く.また,行数と要素数が等しい場合は,新しい列を指定して\texttt{<1d\_ndarray>}や\texttt{<list>}を代入する文を書くことでその配列を追加することができる.

\begin{gram} 
\begin{itemize}
\item \texttt{<DataFrame>[<str>] = a}: \texttt{<DataFrame>}の新しい列\texttt{<str>}の全要素に\texttt{a}を代入する.
\item \texttt{<DataFrame>[<str>] = <1d\_ndarray or list>}: \texttt{<DataFrame>}の新しい列\texttt{<str>}に\texttt{<1d\_ndarray or list>}を代入する(列ベクトルとしての\texttt{<2d\_ndarray>}でも問題ないが,行ベクトルとしての\texttt{<2d\_ndarray>}はエラーとなるので注意).
\end{itemize}
\end{gram}
\begin{cod}[\texttt{pd3.py}] 
\lstinputlisting[backgroundcolor={\color[gray]{.95}}]{code/pd3.py}
\vspace{-10pt}
\begin{lstlisting}
df=
   population   profit  all_1  ndarray  list
0      6.1101  17.5920      1        0     0
1      5.5277   9.1302      1        1     1
2      8.5186  13.6620      1        2     2
3      7.0032  11.8540      1        3     3
4      5.8598   6.8233      1        4     4
\end{lstlisting}
\end{cod}
\vspace{-10pt}

\subsection{\texttt{<DataFrame>}の一部分の取り出し}

トレーニングデータとテストデータに分ける,あるいは特徴量と目的変数に分ける等,機械学習では\texttt{<DataFrame>}から一部分を取り出す操作を頻繁に行う.取り出したものの列が1つしかない場合,\texttt{<DataFrame>}ではなく\texttt{<Series>}で取り出されるので注意.

\begin{rem}
取り出したものの列が1つしかない場合,\texttt{<DataFrame>}ではなく\texttt{<Series>}で取り出される.
\end{rem}

\begin{gram} 
\begin{itemize}
\item \texttt{<DataFrame>[<str>]}: \texttt{<DataFrame>}の列\texttt{<str>}を\texttt{<Series>}で抜き出す.
\item \texttt{<DataFrame>[<list>]}: \texttt{<DataFrame>}の\texttt{<list>}で指定した複数列を\texttt{<DataFrame>}で抜き出す.
\item \texttt{<DataFrame>[a:b:c]}: \texttt{<DataFrame>}のスライス\texttt{a:b:c}で指定した行を\texttt{<DataFrame>}で抜き出す.(1行だけの抜き出しでも\texttt{<Series>}ではなく\texttt{<DataFrame>}となる.)
\item \texttt{<DataFrame>.loc[a:b:c,<str or list>]}: \texttt{<DataFrame>}のスライス\texttt{a:b:c}で指定した行かつ\texttt{<str or list>}で指定した列で選ばれる表データを\texttt{<DataFrame>}で抜き出す.
\end{itemize}
\end{gram}

\begin{cod}[\texttt{pd4.py}] 
\lstinputlisting[backgroundcolor={\color[gray]{.95}}]{code/pd4.py}
\vspace{-10pt}
\begin{lstlisting}
df_sub1=
0    17.5920
1     9.1302
2    13.6620
3    11.8540
4     6.8233
Name: profit, dtype: float64
type=<class 'pandas.core.series.Series'>

df_sub2=
   list  population
0     0      6.1101
1     1      5.5277

2     2      8.5186
3     3      7.0032
4     4      5.8598
type=<class 'pandas.core.frame.DataFrame'>

df_sub3=
   population  profit  all_1  ndarray  list
0      6.1101  17.592      1        0     0
type=<class 'pandas.core.frame.DataFrame'>

df_sub4=
   population  profit  all_1  ndarray  list
2      8.5186  13.662      1        2     2
3      7.0032  11.854      1        3     3
type=<class 'pandas.core.frame.DataFrame'>

df_sub5=
3    3
4    4
Name: list, dtype: int64
type=<class 'pandas.core.series.Series'>

df_sub6=
    profit  ndarray
0  17.5920        0
2  13.6620        2
4   6.8233        4
type=<class 'pandas.core.frame.DataFrame'>

\end{lstlisting}
\end{cod}
\vspace{-10pt}

\subsection{\texttt{<Series>}と1列しか持たない\texttt{<DataFrame>}の違い}
\texttt{<Series>}と\texttt{<DataFrame>}の関係は,\texttt{<1d\_ndarray>}と\texttt{<2d\_ndarray>}と同じである.すなわち,1列しか持たない\texttt{<DataFrame>}と\texttt{<Series>}は\texttt{shape}が異なる.なお,\texttt{<Series>}を\texttt{<DataFrame>}に変換する場合は,\texttt{pd.DataFrame(<Series>)}とすればよい.

\begin{gram} 
\begin{itemize}
\item \texttt{pd.DataFrame(<Series>)}: \texttt{<Series>}を\texttt{<DataFrame>}に変換する.
\end{itemize}
\end{gram}
\begin{cod}[\texttt{pd5.py}] 
\lstinputlisting[backgroundcolor={\color[gray]{.95}}]{code/pd5.py}
\vspace{-10pt}
\begin{lstlisting}
df_sub1=
0    17.5920
1     9.1302
2    13.6620
3    11.8540
4     6.8233
Name: profit, dtype: float64
type=<class 'pandas.core.series.Series'>
shape=(5,)

df_sub2=
   population  profit
2      8.5186  13.662
type=<class 'pandas.core.frame.DataFrame'>
shape=(1, 2)

df_sub3=
    profit
0  17.5920
1   9.1302
2  13.6620
3  11.8540
4   6.8233
type=<class 'pandas.core.frame.DataFrame'>
shape=(5, 1)
\end{lstlisting}
\end{cod}
\vspace{-10pt}

\subsection{\texttt{values}メソッド}
\texttt{<DataFrame or Series>}から,indexや列名を除いた値のみを取り出したいときは\texttt{<DataFrame or Series>.values}メソッドを使う.\texttt{<DataFrame>}の場合は\texttt{<2d\_ndarray>},\texttt{<Series>}の場合は\texttt{<1d\_ndarray>}となる.

\begin{gram} 
\begin{itemize}
\item \texttt{<DataFrame or Series>.values}: \texttt{<DataFrame or Series>}から,indexや列名を除いた値のみを取り出して\texttt{<ndarray>}として格納する.ここで,\texttt{<DataFrame>}の場合は\texttt{<2d\_ndarray>},\texttt{<Series>}の場合は\texttt{<1d\_ndarray>}である.特に,1行しかない\texttt{<DataFrame>}は行ベクトル,1列しかない\texttt{<DataFrame>}は列ベクトルとなることに注意.
\end{itemize}
\end{gram}


\begin{cod}[\texttt{pd2.py}] 
\lstinputlisting[backgroundcolor={\color[gray]{.95}}]{code/pd2.py}
\vspace{-10pt}
\begin{lstlisting}
array_1=
[[ 6.1101 17.592 ]
 [ 5.5277  9.1302]
 [ 8.5186 13.662 ]
 [ 7.0032 11.854 ]
 [ 5.8598  6.8233]]
type=<class 'numpy.ndarray'>
shape=(5, 2)

array_2=
[6.1101 5.5277 8.5186 7.0032 5.8598]
type=<class 'numpy.ndarray'>
shape=(5,)

array_3=
[[ 6.1101 17.592 ]]
type=<class 'numpy.ndarray'>
shape=(1, 2)

array_4=
[[17.592 ]
 [ 9.1302]
 [13.662 ]
 [11.854 ]
 [ 6.8233]]
type=<class 'numpy.ndarray'>
shape=(5, 1)
\end{lstlisting}
\end{cod}
\vspace{-10pt}