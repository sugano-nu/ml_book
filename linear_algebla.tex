\chapter{線形代数}

\section{基本的な結果}

\begin{defi}
$X^T$を{\bf 転置行列}という.	
\end{defi}

\begin{theo}
\begin{align}
{\bm a}^T{\bm b}&= {\bm b}^T{\bm a}	\\
(k{\bm a})^T{\bm b}&= k{\bm a}^T{\bm b} \\
({\bm a}+{\bm b})^T({\bm c}+{\bm d})&= {\bm a}^T{\bm c}+{\bm b}^T{\bm c}+{\bm a}^T{\bm d}+{\bm b}^T{\bm d}
\end{align}
\end{theo}

\begin{theo}
\begin{align}
(A+B)^T&= A^T+B^T \\
(AB)^T&=B^TA^T	
\end{align}
\end{theo}

\begin{defi}
$X^T=X$を満たす行列を{\bf 対称行列}という.	
\end{defi}

\begin{theo}[1次形式,2次形式の勾配ベクトル]
\begin{align}
\nabla_{{\bm x}}({\bm a}^T {\bm x})&=\nabla_{{\bm x}}({\bm x}^T {\bm a})={\bm a}\\
\nabla_{{\bm x}}({\bm x}^T A{\bm x})&=(A+A^T){\bm x}
\end{align}
\end{theo}

\begin{theo}
$X^TX$は対称行列である.
\end{theo}